\chapter{Project Human Resource Management}

Human Resource Management is the proper coordination and organisation of the people within a project. It affects the "way the organisation acquires and uses human resources. and how employees experience the employment relationship" \parencite{hr}

\section{Plan Human Resource Management}

The key output of this stage is a human resource management plan. In order to ensure that a proper plan is developed, it is necessary to be cognisant of a number of areas.

\subsection{Roles}

There are a number of roles needs for the development of this project. The project manager needs to ensure that the right people are in the right roles.

\begin{enumerate}
\item Project Manager
\begin{itemize}
\item Responsible for managing as aspects of the project, such as ensuring the right people are in the right roles.
\end{itemize}
\item Software Architect
\begin{itemize}
\item Responsible for choosing the architecture and language required to implement the project. They are also responsible for the design of the system.
\end{itemize}
\item Designers
\begin{itemize}
\item Responsible for creating the design of the product, such as appearance, art etc.
\end{itemize}
\item Developers
\begin{itemize}
\item Responsible for creating the actual product according the the specification given by the project manager
\end{itemize}
\item Testers
\begin{itemize}
\item Responsible for testing the product produced by the developers and finding errors within it.
\end{itemize}
\end{enumerate}

\subsection{Relationships}

The project manager needs to be aware of previously existing relationships within the project dynamics. These can be both beneficial and detrimental to the project result, and need to be managed to ensure that the project in improved by human interaction, rather than hurt. 

\section{Authority and Organisation}

There needs to be a clear breakdown of who is responsible for what. An example of how to do this is using a responsibility assignment matrix (RAM). Figure~\ref{fig:ram} is an example of a RACI (Responsible, Accountable, Consult, and Inform) chart, which shows with role is responsible for what actions within the scope of this project.

\begin{table}[H]
\caption{RACI Matrix, adapted from \parencite{pmbok}}
\begin{center}
    \begin{tabular}{ | l | l | l | l | l | l | p{5cm} |}
    \hline
	\textbf{Activity} & \textbf{Project Manager} & \textbf{Architect} & \textbf{Designer} & \textbf{Developer} & \textbf{Tester}\\ \hline
	\textbf{Create charter} & R & I & I & I & I \\ \hline
	\textbf{Collect Requirements} & I & A & R & I & I \\ \hline
	\textbf{Submit Change Request} & I & A & R & I & I \\ \hline
	\textbf{Develop Test Plan} & A & I & C & I & R \\ \hline
    \end{tabular}
	A - Accountable, R - Responsible, I - Informed, C - Consult

\end{center}
\label{fig:ram}
\end{table}

\section{Identify Competencies}

This is the use of "expert judgement" \parencite{pmbok} by the project manager. The PMBOK provides a number of areas for the project manager to focus on:

\begin{enumerate}
\item List the preliminary requirements for the required skills; 
\item Assess the roles required for the project based on standardized role descriptions within the organization;
\item Determine the preliminary effort level and number of resources needed to meet project objectives;
\item Determine reporting relationships needed based on the organizational culture;
\item Provide guidelines on lead time required for staffing, based on lessons learned and market conditions;
\item Identify risks associated with staff acquisition, retention, and release plans; and
\item Identify and recommend programs for complying with applicable government and union contracts.
\begin{itemize}
\item \parencite{pmbok}
\end{itemize}
\end{enumerate}

\section{Organisation and Planning}

In the present day, the concept of Global Software Development is widely explored as a method for software development in many companies. As such, project managers need to react to this change in the working environment. There is a "need to capitalize on the global resource pool to successfully and cost competitively use scarce resources" \parencite{gsd}. With the diversity introduced by this development, "close cooperation of individuals with different cultural backgrounds" \parencite{gsd} is required by a project manager.

The PMBOK technique, Virtual Teams, would be beneficial in a project such as this. A Virtual Team is a group of people with a shared goal who fulfil their roles within the project, without having to actually meet face to face. Communication is facilitated through email, telephone calls, with work being managed by a source control system.



