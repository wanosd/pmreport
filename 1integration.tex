\chapter{Project Integration}

Project Integration Management includes the processes and activities to identify, define, combine, unify, and coordinate the various processes and project management activities within the Project Management Process Groups. In the project management context, integration includes characteristics of unification, consolidation, communication, and integrative actions that are crucial to controlled project execution through completion, successfully managing stakeholder expectations, and meeting requirements. \parencite{pmbok}

\section{Develop Project Charter}

The project charter is in basic terms similar to a contract for a project, however, unlike a contract there is no financial data in the charter. It is formal acknowledgement of the project allowing an organisation to incorporate the project into their plans.
At this point a statement of work has been submitted as well as a business case describing the tennis club project, these are both created by the project sponsor which in this case is the tennis club associated with the project. The project charter is created by the tennis club in collaboration with our project manager, this ensures that the goals set are realistic and can be completed by our company. We have created a formal relationship with the club and now that a charter has been created we can move on to planning the project. Using this project charter we can ensure the product ships on time.
We will also use organisational templates from the organisational process assets inputted to the charter to create a template for the project.
The project charter is as follows:

\begin{enumerate}
\item \textbf{Project purpose or justification}
\begin{itemize}
\item The project is to build a tennis club website for a local club.
\end{itemize}
\item \textbf{Measurable project objectives and related success criteria}
\begin{itemize}
\item A finished product that is usable and reliable for club members and is better than current paper based system.
\end{itemize}
\item \textbf{High-level requirements}
\begin{itemize}
\item General tennis club website for hosing club details and contact information.
\item Member system for registering and paying fees
\item System to book courts
\item Tournament automation
\item News blog
\end{itemize}
\item \textbf{Assumptions and constraints}
\begin{itemize}
\item The stakeholder assumes that developers understand the domain of club websites and can complete the functionality required. Stakeholder also assumes that the company can provide future contract work to update the server operating system, middle ware and make any updates that are required to the core product.

\item Constraints are a limited budget as this is a community club and not a full time company.
\end{itemize}

\item \textbf{High-level risks}
\begin{itemize}
\item In the project high level risks are security and data protection. If the club members personal data, passwords or payment information was compromised it could risk the integrity of the club.
\end{itemize}

\item \textbf{Summary milestone schedule}
Project milestones include:
\begin{itemize}
\item Finishing the planning stage of the project (1 week)
\item Allocating resources to the project (1 day)
\item Development of the project (6 months)
\item Testing (1 month)
\item Deployment (1 week)
\end{itemize}

\item \textbf{Summary budget}
\begin{itemize}
\item The total budget for the project is €50,000
\end{itemize}

\item \textbf{Stakeholder list}
\begin{itemize}
\item Tennis club committee
\item Tennis club members
\item Project manager
\item Developers
\end{itemize}

\item \textbf{Project approval requirements}
\begin{itemize}
\item The project will be deemed a success if the website is operational for one month without issue. In general terms the tennis club makes the decision on if the project is a success and they sign off on project completion.
\end{itemize}

\item \textbf{Assigned project manager, responsibility, and authority level}
\begin{itemize}
\item Joe Blogs will be the project manager who assumes responsibility for this project in total. He will manage the test and development teams and report their progress to stakeholders.
\end{itemize}
\item \textbf{Name and authority of the sponsor or other person(s) authorizing the project charter}
\begin{itemize}
\item John Smith, Club captain.
\end{itemize}
\end{enumerate}

\subsection{Develop Project Management Plan}

The project management plan defines how the project is executed, monitored and controlled, and closed. The project management plan’s content varies depending upon the application area and complexity of the project \parencite{pmbok}.

This overall document serves as a project management plan.

\subsection{Directing and Managing Project Work}

Direct and Manage Project Work is the process of leading and performing the work defined in the project management plan and implementing approved changes to achieve the project’s objectives \parencite{pmbok}. Obviously this process is managed by the project manager.

Once the project plan is complete the concrete work on the project can begin. Directing and managing the project work allows for corrective and preventive action as well as defect repair. The project management information system that will be used in the project is Microsoft Project. Atlassian Jira will also be used to track tasks, defects and blocking issues during project construction.

\subsection{Monitoring and Controlling Project Work}

Monitor and Control Project Work is the process of tracking, reviewing, and reporting the progress to meet the performance objectives defined in the project management plan. The key benefit of this process is that it allows stakeholders to understand the current state of the project, the steps taken, and budget, schedule, and scope forecasts \parencite{pmbok}. The tools established in section 2.3 will be used to monitor and control the project, as well as the following techniques

\begin{enumerate}
\item Regression analysis
\item Causal analysis
\item Root cause analysis
\item Reserve analysis
\item Trend analysis
\item Earned value management
\item Variance analysis
\end{enumerate}

Weekly meetings discussed elsewhere in this project management plan will be held to monitor and control the project.

\subsection{Integrated Change Control}

Perform Integrated Change Control is the process of reviewing all change requests; approving changes and managing changes to deliverables, organizational process assets, project documents, and the project management plan; and communicating their disposition. It reviews all requests for changes or modifications to project documents, deliverables, baselines, or the project management plan and approves or rejects the changes. The key benefit of this process is that it allows for documented changes within the project to be considered in an integrated fashion while reducing project risk, which often arises from changes made without consideration to the overall project objectives or plans \parencite{pmbok}.

Changes to the project scope or definition can be made through the project manager who will approve the change with the relevant stakeholder in stakeholder meetings, thereby mitigating risk. The changes to the project scope will be documented in change control tools so that there is an audit trail for changes made to the project whether the changes are accepted or not.


\subsection{Close}

Close Project or Phase is the process of finalizing all activities across all of the Project Management Process Groups to formally complete the project or phase. The key benefit of this process is that it provides lessons learned, the formal ending of project work, and the release of organization resources to pursue new endeavours \parencite{pmbok}.

The project manager will complete a review of the tennis club website and declare it completed or not. A scope baseline review will also be completed. Once all features are completed the project will be handed over to the deployment team who will deploy the web application to a web server and test it for functionality and load balancing.
